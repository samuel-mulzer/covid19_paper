\documentclass[../main.tex]{subfiles}

\begin{document}
    Die Ausbreitung von Covid-19 in Deutschland stellt die Gesellschaft seit dem Frühjahr 2020 vor beispiellose Herausforderungen. Insbesondere in modernen vernetzten Gesellschaften mit hoher Bevölkerungsdichte stellen leicht übertragbare Infektionskrankheiten eine Gefahr für die Gesundheitsversorgung dar, deren Aufrechterhaltung stets erklärtes Ziel war.
    In diesem Zusammenhang kommt der Epidemiologie eine wichtige Rolle zu. Sie liefert eine Vorhersage der zukünftigen Entwicklung, ermöglicht Planung und bietet Politik und Gesundheitsbehörden eine Grundlage für ihre Entscheidungen.

    Mittlerweile stehen wirksame Impfstoffe zur Verfügung und auch durch die saisonale Abschwächung der Kontagiosität war im Sommer eine Entspannung der Lage erkennbar. Daher ist nun die langfristige Entwicklung der Pandemie von Interesse.
    Es soll untersucht werden, ob es durch Covid-19 in Deutschland zu weiteren Wellen kommt, oder ob die Krankheit bereits endemisch wird.
    Insbesondere stellt sich die Frage, ob in diesem Winter ein erneuter Anstieg der Infektionszahlen zu erwarten ist, der zu einer Überlastung des Gesundheitssystems führen kann.

    Dafür verwendet die Epidemiologie mathematische Modelle, die auf dem aktuellen biologischen Wissen über den Erreger basieren. Diese Modelle werden analysiert, um mögliche Verläufe zu identifizieren. Des Weiteren wird das Modell überprüft, indem Simulationen mit realen Daten abgeglichen werden. Wenn nötig werden die zugrundeliegenden Annahmen angepasst und der Kreislauf beginnt von neuem.

    Zunächst wird das klassische SIR-Modell betrachtet, um Konzepte wie die Reproduktionszahl oder den Schwellenwert einer Epidemie herzuleiten. Da das SIR-Modell essentielle Einflussfaktoren vernachlässigt und den realen Verlauf daher unzureichend beschreibt, wird ein erweitertes Modell erstellt, das Immunitätsverlust, Impfungen und saisonale Schwankungen miteinbezieht. Mit diesem werden mehrere Szenarien simuliert, um den Einfluss verschiedener Parameter auf die weitere Entwicklung von Covid-19 in Deutschland zu untersuchen.
\end{document}
