\documentclass[../main.tex]{subfiles}

\begin{document}
    In dieser Arbeit wurde die weitere Entwicklung von Covid-19 untersucht. Da sich das klassische SIR-Modell als nicht aussagekräftig erwiesen hat, wurde ein erweitertes Modell entwickelt. Dabei handelt es sich um ein SIRS-Modell mit Impfungen und Saisonalität, das zur Simulation verschiedener Szenarien verwendet wurde.
    Die Ergebnisse der Simulation deuten auf das Eintreten einer weiteren Welle im Winter 2022/23 und eine anschließende Entwicklung zur Endemie hin. Dabei wäre eine Überlastung des Gesundheitssystems zu erwarten, wenn keine Eindämmungsmaßnahmen getroffen werden. Dies kann jedoch durch ein höheres Impftempo und Maßnahmen der Kontaktreduzierung verhindert werden. Im weiteren Verlauf kommt es als Folge der Saisonalität jährlich im Winter zu einer Welle mit konstantem Ausmaß.
    Außerdem hat sich gezeigt, dass Impfungen zur langfristigen Kontrolle von Covid-19 entscheidend sind. Maßnahmen der Kontaktreduzierung können das Ausmaß der bevorstehenden Winterwelle zudem verzögern, sind auf lange Sicht aber nicht effektiv.

    Diese Ergebnisse sind jedoch nicht endgültig und müssen aus mehreren Gründen hinterfragt werden. Zunächst wurden vereinfachende Annahmen wie die einer homogenen Bevölkerungsstruktur getroffen, welche die Aussagekraft des Modells einschränken.
    Zudem bestehen Unsicherheiten hinsichtlich des biologischen Wissens über Covid-19. Dies betrifft vor allem den Umfang und die Dauer der Immunität. Des Weiteren kann es durch das Auftreten einer neuen immunevasiven Mutante, welche Omikron verdrängt, zu einer gänzlich neuen Situation kommen.
    Die genauere Untersuchung der Krankheit und die Entwicklung realistischerer Modelle stellen daher Ansatzpunkte für weitere Forschung dar.

    Sicher lässt sich jedoch sagen, dass Covid-19 auch in Zukunft aktuell bleiben wird, sowie dass die von dem Virus ausgehende Bedrohung mittlerweile kontrollierbar geworden ist.
\end{document}