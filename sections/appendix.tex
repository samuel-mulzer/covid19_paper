\documentclass[../main.tex]{subfiles}

\begin{document}
    \subsection*{Normalisierung des SIR-Modells}
    \label{ap:normalization}
    Nach Ableitung mit der Quotientenregel ergibt sich mit $N' = 0$ allgemein:
    \begin{equation*}
        x' = \left(\frac{X}{N}\right)' = \frac{X'N-N'X}{N^2} = \frac{X'N}{N^2} = \frac{X'}{N}
    \end{equation*}
    und entsprechend:
    \begin{equation*}
        \begin{aligned}
            s' &= \frac{S'}{N}, & i' &= \frac{I'}{N}, & r' &= \frac{R'}{N}
        \end{aligned}
    \end{equation*}
    In diese Ausdrücke kann nun jeweils die rechte Seite aus System \eqref{eq:sir} eingesetzt werden und wir erhalten: 
    \begin{equation*}
        \setlength{\jot}{0.5cm}
        \begin{aligned}
            s' &= \frac{-\frac{\beta}{N}SI}{N} = \frac{\beta SI}{N^2} = -\beta\frac{S}{N}\frac{I}{N} = -\beta si, \\
            i' &= \frac{-\frac{\beta}{N}SI-\gamma I}{N} = \frac{\beta SI}{N^2} - \frac{\gamma I}{N} = \beta\frac{S}{N}\frac{I}{N} -\gamma\frac{I}{N} = \beta si - \gamma i, \\
            r' &= \frac{\gamma I}{N} = \gamma\frac{I}{N} = \gamma i
        \end{aligned}
    \end{equation*}
    Nachdem die Brüche nun wieder durch die Anteile aus \eqref{eq:proportions} ersetzt wurden, erhalten wir System \eqref{eq:sir_norm}.
    
    
    \subsection*{Software zur Simulation der Modelle}
    \label{ap:simulation}
    Im Rahmen der Arbeit wurde Software entwickelt, mit der  die verwendeten Modelle simuliert wurden. Diese ist in Python geschrieben und kann in einer mit Cython optimierten Variante abgerufen werden: \url{https://github.com/samuel-mulzer/covid19_model}
    Im Folgenden finden sich vereinfachte Versionen der einzelnen Bestandteile:

\begin{lstlisting}[title=Importierung der verwendeten Programmbibliotheken:]
from math import sin, pi
import numpy as np
import matplotlib.pyplot as plt
\end{lstlisting}

\begin{lstlisting}[title=Definition der Differentialgleichungssysteme der verwendeten Modelle:]
# SIR model
def sir(x, t, beta, gamma):
    s,i,r = x

    dsdt = - beta * s * i
    didt = + beta * s * i - gamma * i
    drdt = + gamma * i

    return np.array([dsdt, didt, drdt])

# SIRS model with vaccination and seasonality
def sirs_vs(x, t, beta, gamma, alpha, zeta, xi):
    s,i,r = x

    _lambda = beta * (1 + xi * sin(pi*t/180))
    _eta    = zeta if s > zeta else s

    dsdt = - _lambda * s * i - _eta + alpha * r
    didt = + _lambda * s * i - gamma * i
    drdt = + gamma * i + _eta - alpha * r 

    return np.array([dsdt, didt, drdt])
\end{lstlisting}

\begin{lstlisting}[title=Lösung des Systems mit dem Euler-Verfahren:]
def solve(model, x0, tspan, h, args):
    solution = np.empty(shape=(len(tspan), len(x0)))
    c = 0
    solution[c] = x0
    x = x0
    
    for t in tspan[1:]:
        c += 1
        dx = model(x, t, *args) * h
        x += dx
        solution[c] = x

    return solution.transpose()
\end{lstlisting}

\begin{lstlisting}[title=Visualisierung:]
def visualize(solution, n, tspan, args, labels, save):
    t_0, t_max = tspan[0], tspan[-1]
    beta, gamma = args[0], args[1]

    fig = plt.figure(figsize=(12,4), constrained_layout=True)
    spec = fig.add_gridspec(2,2, width_ratios=[2,1], height_ratios=[3,2])

    ax1 = fig.add_subplot(spec[0, 0])
    ax2 = fig.add_subplot(spec[1, 0])
    ax3 = fig.add_subplot(spec[:, 1])

    c = 0
    for x in solution:
        ax1.plot(tspan, x, '-', label=labels[c])
        c += 1
    ax1.legend(loc='center right')
    ax1.set_xlim(t_0, t_max)
    ax1.set_ylim(0, 1)

    reprod = beta/gamma*solution[0]
    ax2.plot(tspan, reprod, 'k', label=r"$R_e$")
    ax2.plot([t_0, t_max], [1,1], 'k:')
    ax2.legend()
    ax2.set_xlim(t_0, t_max)

    ax3.plot(solution[0], solution[1], 'k')
    ax3.set_xlim(0, None)
    ax3.set_ylim(0, None)

    plt.show()
\end{lstlisting}

\begin{lstlisting}[title=Simulation des Standardszenarios des erweiterten Modells:]
t = 0
d = 1000
h = 1
tspan = np.arange(t, t+d, h)

s0, i0, r0 = 0.308, 0.007, 0.685
n = 83e6
x0 = np.array([s0, i0, r0])
labels = ["S", "I", "R"]

beta = 0.95
gamma = 1/10
alpha = 1/90
zeta = 0.0015
xi = 0.46
args = (beta, gamma, alpha, zeta, xi)

solution = solve(sirs_vs, x0, tspan, h, args)
visualize(solution, n, tspan, args, labels)

index_i_max = solution[1].argmax()
i_max = solution[1][index_i_max]
print(f"i_max={i_max:.2g}, I_max={i_max*n:.3g} | t={tspan[index_i_max]}")
\end{lstlisting}



    \subsection*{Software zur Auswertung der Datensätze}
    \label{ap:evaluation}
    Die Auswertung der Datensätze erfolgt ebenfalls in Python und ist in Form eines Jupyter Notebooks Teil des zuvor erwähnten Softwareprojekts.

\begin{lstlisting}[title=Importierung der verwendeten Programmbibliotheken:]
import pandas as pd
import statistics as st
\end{lstlisting}

\begin{lstlisting}[title=Definition von Konstanten:]
p = 83e6    # Bevölkerung Deutschlands
d1 = 10     # Dauer der Infektiosität
d2 = 180    # Dauer der Immunität
\end{lstlisting}

\begin{lstlisting}[title=Laden der Datensätze:]
# Neuinfektionen
df1 = pd.read_csv("https://raw.githubusercontent.com/robert-koch-institut/SARS-CoV-2-Nowcasting_und_-R-Schaetzung/main/Nowcast_R_aktuell.csv")
df1 = df1[["Datum", "PS_COVID_Faelle_ma4"]]
df1.rename(inplace=True, columns={"Datum": "date", "PS_COVID_Faelle_ma4": "cases"})

# Impfungen
df2 = pd.read_csv("https://impfdashboard.de/static/data/germany_vaccinations_timeseries_v3.tsv", sep="\t")
df2 = df2[["date", "impfungen","impfquote_boost1"]]
df2.rename(inplace=True, columns={"date": "date", "impfungen": "vaccinations", "impfquote_boost1": "boost"})

# Intensivbelegung
df3 = pd.read_csv("https://diviexchange.blob.core.windows.net/%24web/zeitreihe-deutschland.csv")
df3 = df3.loc[df3["Behandlungsgruppe"] == "ERWACHSENE"]
df3 = df3[["Datum", "Aktuelle_COVID_Faelle_ITS", "Belegte_Intensivbetten", "Freie_Intensivbetten"]]
df3.rename(inplace=True, columns={"Datum":"date","Aktuelle_COVID_Faelle_ITS":"covid","Belegte_Intensivbetten":"occupied","Freie_Intensivbetten":"free"})
df3.reset_index(drop=True, inplace=True)
\end{lstlisting}

\begin{lstlisting}[title=Bestimmung des Anfangszustands:]
t0 = df1.loc[df1["date"] == "2022-09-30"].index[0]
i = sum(df1["cases"][t0-d1:])       # infected
r1 = sum(df1["cases"][t0-d2:t0-d1]) # recovered
t0 = df2.loc[df2["date"] == "2022-09-30"].index[0]
r2 = df2["boost"].iloc[t0]          # boosted

i *= 2/p
r = r1/p + r2
s = 1 - i - r

print(s, i, r)
\end{lstlisting}

\begin{lstlisting}[title=Bestimmung der Impfrate:]
# Durschnittliche tägliche Impfungen 2022
t0 = df2.loc[df2["date"] == "2022-01-01"].index[0]
t1 = df2.loc[df2["date"] == "2022-09-30"].index[0]
v = st.mean(df2["vaccinations"][t0:t1])
print(v)
\end{lstlisting}

\begin{lstlisting}[title=Bestimmung der Kapazität des Gesundheitssystems:]
# Maximale Anzahl von auf der Intensivstation behandelten Covid-19 Patienten
index_icu_max = df3["covid"].idxmax()
icu_max = df3["covid"].iloc[index_icu_max]
# Maximale Anzahl von Covid-19 Infizierten
i_max = icu_max / 0.003 / 0.13
print(icu_max, i_max)
\end{lstlisting}

\end{document}